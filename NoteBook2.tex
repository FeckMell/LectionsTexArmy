\documentclass[12pt,oneside]{article}
\usepackage[utf8]{inputenc}
\usepackage[english,russian]{babel}
\usepackage{enumerate}
\usepackage{indentfirst}
\usepackage{misccorr}
\usepackage{amsmath}
\usepackage{amssymb}
\usepackage{amsthm}
\usepackage{amsfonts}
\usepackage{mathtools}
\usepackage{color}
\usepackage{cmap}
\usepackage{titlesec}
\usepackage[dvips]{graphicx}
%\graphicspath{C:\Users\dm\Desktop\Диплом}
\usepackage{listings}




%\renewcommand{\abstractname}{}
\newtheorem{theorem}{Теорема}[section]
\newtheorem{hypoth}[theorem]{Гипотеза}
\newtheorem{lemma}[theorem]{Лемма}
\newtheorem{determ}[theorem]{Определение}
\newtheorem{naming}[theorem]{Обозначение}
\newtheorem{ffff}[theorem]{Замечание}
\newtheorem{sentence}[theorem]{Предложение}
\newtheorem{iwantthat}[theorem]{Утверждение}
\newtheorem{fromit}[theorem]{Следствие}
\theoremstyle{definition}
\newtheorem{example}[theorem]{Пример}
\newtheorem{task}[theorem]{Задача}
\setcounter{MaxMatrixCols}{50}
%\DeclarePairedDelimiter{\norm}{\lVert}{\rVert}

\newcommand{\ZZ}{\mathbb{Z}}
\newcommand{\NN}{\mathbb{N}}
\newcommand{\QQ}{\mathbb{Q}}
\newcommand{\RR}{\mathbb{R}}


\begin{document}




\tableofcontents


%\mathcal{P}
%\mathbb{N}
%
%
%
\newpage
\section{Основные алгебраические структуры}\label{ch_basic_alg_st}
\subsection{Бинарные операции и их свойства}

\begin{determ}
Пусть $A$ - множество. Бинарной операцией на множестве $A$ называется отображение:
\begin{equation}
f:A^2 \rightarrow A \quad (f: A \times A \rightarrow A)
\end{equation}
\end{determ}

\begin{ffff}
	Если $f$ - бинарная операция на $A$ и пара $(a,b) \in A^2$, то образ пары $(a,b)$ при отображении $f$ называется значением операции $f$ на элементах $a$ и $b$ (результатом применения операции $f$ к элементам $a$ и $b$) и обозначается $f(a,b)$ или $afb$.
\end{ffff}

\begin{example}\label{bin_examples}
	Примеры бинарных операций
	\begin{enumerate}
		\item Сложение и усножение на множествах $\NN, \NN_0, \ZZ, \QQ, \RR$.
		\item Вычитание на $\ZZ, \QQ, \RR$ - определеноб $\NN, \NN_0$ - не определено.
		\item Пусть $f_1, f_2$ такие, что $f_1:(1,n)^2 \rightarrow (\overline{1,n})$, $f_2:(\overline{1,n})^2 \rightarrow(1,n)$ при этом
		\begin{equation}
			f_1(a,b) = \max\{a,b\},\quad f_2(a,b) = \min\{a,b\}
		\end{equation}\\
		Так как $\forall a,b \in \overline{1,n}$ $\max$ и $\min$ однозначно определены и содержатся во множестве от $\overline{1,n}$, то отображения $f_1, f_2$ являются бинарными операциями на множестве $\overline{1,n}$.
		\item $M$ - множество, $P(M)$ - множество всех подмножеств, тогда пересечение и объединение $\forall A,B \in P(M)$ является бинарными операциями на множестве $P(M)$. 
	\end{enumerate}
\end{example}

\begin{determ}
	Бинарная операция $*$ на множестве $M$ называется ассоциативной, если $\forall a,b \in M$ выполняется условие $(a*b)*c = a*(b*c)$.\\
	Примерами ассоциативных операций могут служить бинарные операции из примера  \ref{bin_examples}.
\end{determ}

\begin{determ}
	Бинарная операция $*$ на множестве $M$ называется коммутативной, если $\forall a,b \in M$ выполняется условие
	\begin{equation}
		a*b=b*a
	\end{equation}
\end{determ}

\begin{example}
	Примеры коммутативных и некоммутативных операций:
	\begin{enumerate}
		\item Коммутативные - пункты $1, 3, 4$ из \ref{bin_examples}.
		\item Некоммутативная - пункт $2$ из \ref{bin_examples}, декартово произведение, композиция.
	\end{enumerate}
\end{example}

\begin{ffff}
	Для отдельных элементов $a,b \in M$ равенство $a*b=b*a$ может выполняться в том случае, если операция $*$ не коммутативна. Такие элементы называются перестановочными (коммутирующими) друг с другом
\end{ffff}

\begin{example}
	$a=0, b=0: \quad a-b=b-a$
\end{example}

\begin{ffff}
	Свойства ассоциативности и коммутативности операции независимы. пример коммутативной но не ассоциативной операции:
	\begin{equation}
	a*b=\frac{a+b}{2}
	\end{equation}
\end{ffff}

\begin{determ}
	Бинарная операция $*$ на множестве $M$ называется леводистрибутивной(праводистрибутивной) относительно операции $o$ если $\forall$ $a,b,c \in M$ выполнено условие:
	\begin{gather}
	a*(b \circ c) = (a*b)\circ(a*c) \quad\mbox{ - леводистрибутивная}\\
	(b \circ c)*a = (b*a)\circ(c*a) \quad\mbox{ - праводистрибутивная}
	\end{gather}
	если выполняются оба этих равенства, то говорят, что $*$ дистрибутивна относительно операции $\circ$. Например умножение дистрибутивна к сложению.
\end{determ}













































\subsection{Алгебраические структуры с одной бинарной операцией}
\begin{determ}
	Алгебраической структурой (алгеброй) называется множество с системой операций.
\end{determ}

\begin{determ}
	Множество $G$ с одной бинарной операцией называют группоидом, обозначают $(G,*)$.
\end{determ}

\begin{ffff}
	Из определения группоида следует, что если множество $G$ - конечно, то правило по которому можно найти значение операции $*$. $\forall a,b \in G$ , можно записать в таблицу $G=\{a_1, \dots, a_n\}$ - т.к. $G$  - конечно.\\
	$\begin{pmatrix}
		* & a_1 & \dots & a_j & \dots & a_n \\
		a_1& a_1*a_1 & \dots & a_1*a_j & \dots & a_1*a_n \\
		\dots &\dots &\dots &\dots &\dots &\dots \\
		a_i& a_i*a_1 & \dots & a_i*a_j & \dots & a_i*a_n \\
		\dots &\dots &\dots &\dots &\dots &\dots \\
		a_n& a_n*a_1 & \dots & a_n*a_j & \dots & a_n*a_n \\
		
	\end{pmatrix}$
\end{ffff}

\begin{determ}
	Пусть $G_1 \subset G$, $\exists (G,*),$ $G_1$ называют замкнутым относительно операции $*$, если выполнены условия $\forall a,b \in G: ab\in G_1$.\\
	В этом случае группоид $(G_1,*)$ называют подгруппоидом группоида $(G,*)$.
\end{determ}

\begin{determ}
	Элемент $\Lambda$ группоида $(G,*)$ называют нейтральным, если $\forall a \in G$ выполнено:\\
	\begin{equation}
	\Lambda*a=a
	\end{equation}
\end{determ}

\begin{example}
	\begin{enumerate}[]
		\item $(\NN_0,+)(\QQ,+) \qquad 0$-нейтральные
		\item $(\NN_0,\bullet)(\QQ,\bullet) \qquad 1$-нейтральные
		\item $(\NN,+)(\ZZ,+) \qquad $не имеют нейтрального элемента
	\end{enumerate}
\end{example}

\begin{iwantthat}
	Если в группоиде $(G,*)$ существует нейтральный элемент, то он единственный.
\end{iwantthat}
\begin{proof}
	пусть это не так, тогда $\Lambda_1, \Lambda_2$ - нейтральные элементы группоида $(G,*)$. Т.к. $\Lambda_1$ - нейтральный, то $\Lambda_1*\Lambda_2=\Lambda_2$, а т.к. $\Lambda_2$ - нейтральный, то $\Lambda_1*\Lambda_2=\Lambda_1$.\\
	Тогда $\Lambda_1=\Lambda_2$.
\end{proof}

\begin{determ}
	Пусть есть $(G,*)$, $\Lambda$ - нейтральный элемент. Элемент $a'\in G$ Называется симметричным элементом элемента $a \in G$, если выполнено условие:
	\begin{equation}
		a'*a=a*a'=\Lambda
	\end{equation}
\end{determ}

\begin{ffff}
	В общем случае в группоиде с нейтральным элементом $\Lambda$ элемент $\alpha$ может не иметь симметричных элементов, а может иметь $1$ или несколько симметричных элементов.
\end{ffff}

\begin{determ}
	Группоид $(G,*)$ с ассоциативной операцией называют полугруппой.
\end{determ}

\begin{iwantthat}
	Если в полугруппе $(G,*)$ с нейтральным элементом $\Lambda$ для элемента $\alpha$ существует симметричный элемент, то он единственный.
\end{iwantthat}
\begin{proof}
	Пусть есть $\alpha'$ и $\alpha''$ - симметричные элементы для $\alpha$, тогда получается:
	\begin{equation}
	\alpha'=\alpha'*\Lambda=\alpha'*(\alpha*\alpha'')=(\alpha'*\alpha)*\alpha''=\Lambda*\alpha''=\alpha'' \Rightarrow \alpha'=\alpha''
	\end{equation}
\end{proof}

\begin{determ}
	Группоид $(G,*)$ называется группой, если выполнены условия:
	\begin{enumerate}
		\item $*$ - ассоциативна
		\item в $(G,*)$ существует нейтральный элемент $\Lambda$
		\item $\forall \alpha \in G \qquad \exists \alpha' \in G$\\
		если кроме того выполнено:
		\item $*$ - коммутативна,\\ то такую группу будут называть абалевой группой
	\end{enumerate}
\end{determ}

\begin{example}
	$(\ZZ,+), (\QQ,+), (\RR,+)$ - группы (абелева группа)\\
	$(\QQ,\bullet), (\RR, \bullet)$ - коммутативные полугруппы с нейтральным элементом $\Lambda=1$, но они не являются группами.\\
	$(\QQ \backslash \{0\}, \bullet), (\RR \backslash \{0\}, \bullet)$ - группы\\
	$(\{1\},\bullet)$ - группа\\
	$(\{1,-1\},\bullet)$ - группа\\
\end{example}

\begin{theorem}
	В любой группе $(G,*) \qquad \forall a,b \in G$ однозначно разрешимы уравнения:\\
	\begin{equation}
		a*x=b \qquad y*a=b
	\end{equation}
\end{theorem}
\begin{proof}
	С помощью непосредственной проверки, можно убедиться, что решением уравнения:\\
	$a*x=b$ является $x=a'*b \rightarrow a*(a'*b)=b$, \\
	и  решением уравнения:\\
	$y*a=b$ является $y=b*a' \rightarrow (a*a')*b=b$\\
	
	Теперь необходимо доказать единственность этих решений. Допустим первое уравнение имеет $2$ решения $x_1, x_2$, тогда:\\
	$a*x_1=a*x_2$ - умножим на $a'$\\
	$a'*(a*x_1)=a'*(a*x_2) \Rightarrow (a'*a)*x_1=(a'*a)*x_2$\\
	$\Lambda*x_1=\Lambda *x_2 \Rightarrow x_1=x_2$ - противоречие, $\Rightarrow$ существует только одно решение. Аналогично с $y$.	
\end{proof}

\begin{determ}
	Пусть есть $(G,*)$ - полугруппа $(a_1,...,a_n)\in G$, если $a_1=a_2=..=a_n$. Тогда
	\begin{enumerate}
		\item  $a_1*a_2*...*a_n=a^n$, если $*$ - умножение.
		\item $a_1*a_2*...*a_n=na$, если $*$ - сложение.
	\end{enumerate}
	В таком случае элемент $a^n$ называется $n$-степенью элемента $a$, элемент $na$ называется $n$-кратным элементом $a$.
\end{determ}

\begin{iwantthat}
	Если $(G,\bullet)$ и $(G,+)$ - полугруппы, то для $\forall a \in G, \forall n_1, n_2 \in \NN$ выполнены условия:
	\begin{enumerate}
		\item $a^{n_1}*a^{n_2}=a^{n_1+n_2}$
		\item $(a^{n_1})^{n_2}=a^{n_1*n_2}$
		\item $n_1a+n_2a=(n_1+n_2)a$
		\item $n_1(n_2a)=(n_1n_2)a$
	\end{enumerate}
\end{iwantthat}

\begin{ffff}
	Если группоид $(G,\bullet) и (G,+)$ являктся группой, то понятия $n$-ой степени и $n$-кратного элемента можно распространить на любое $n\in \ZZ$ для этого введем следующие обозначения:
	\begin{enumerate}
		\item $0$ - нейтральный элемент относительно $+$
		\item $e$ - нейтральный элемент относительно $\bullet$
		\item $-a$ - противоположный элемент к $a$ относительно $+$
		\item $a^{-1}$ - обратный к $a$ относительно $\bullet$
		\item $a^0=e$
		\item $0*a=0$
		\item $(a^n)^{-1}=a^{-n}$
		\item $(-n)a=-(na)$
	\end{enumerate}
\end{ffff}













































\subsection{Кольца и поля}
\begin{determ}
	Кольцом называется множество $R$ с бинарными операциями $ +, \bullet $ если выполены условия:
	\begin{enumerate}
		\item $ (R,+) $ - абелева группа 
		\item $ (R,\bullet) $ - полугруппа
		\item умножение дистрибутивно относительно $+$
	\end{enumerate}
	при этом группа $(R,+)$ называется аддитивной группой кольца $R$.
\end{determ}

\begin{determ}
	Кольцо $(R,+,\bullet)$ называется коммутативным, если умножение коммутативно и кольцом с единицей, если $(R,\bullet)$ - полугруппа с единицей.
\end{determ}

\begin{example}
	\begin{enumerate}[]
		\item $(\ZZ,+,\bullet), (\QQ,+,\bullet), (\RR,+,\bullet)$ - коммутативные кольца с единицей
		\item $(2\ZZ,+,\bullet)$, где $2\ZZ$ - множество всех четных чисел
		\item $R^2=\{(a,b)|a,b\in\RR\}$ - множество упорядоченных пар (кольцо не коммутативно)
	\end{enumerate}
	Введем на множестве $R^2$ операции сложения и умножения:
	\begin{equation}
	\forall(c,d)\in R^2: (a,b)+(c,d) = (a+c,b+d) \qquad (a,b)\bullet(c,d)=(ac,bd)
	\end{equation}
	Так как операции над парами производятся покомпонентно, то из свойств целых чисел получаем:
	\begin{enumerate}
		\item $+,\bullet$ в $R^2$ - коммутативны и ассоциативны
		\item $ \bullet $ - дистрибутивна относительно $+$
		\item $(0,0)$ - нулевой элемент
		\item $(1,1)$ - единичный элемент
		\item $(-a,-b)$ - противоположный элемент для $(a,b)$
	\end{enumerate}
	$(R^2,+,\bullet)$ - коммутативное кольцо с единицей
\end{example}

\begin{theorem}
	$\forall a,b,c \in R^2$, где $R^2$ - произвольное кольцо с нулем, справедливы следующие выражения:
	\begin{enumerate}
		\item\label{e_a} $a*0=0*a=0$
		\item\label{e_b} $-(-a)=a$
		\item\label{e_c} $(-a)b=-(ab)$
		\item\label{e_d} $(-a)(-b)=ab$
		\item\label{e_e} $a(b-c)=ab-ac$
		\item\label{e_f} $m(ab)=(ma)b, \quad m \in \ZZ$
		\item\label{e_g} $(m_1a)(m_2b)=(m_1m_2)(ab), \quad m_1,m_2 \in \ZZ$
	\end{enumerate}
\end{theorem}
\begin{proof}
	\underline{Пункт \ref{e_a}}: $a*0=0*a=0, \quad 0-\Lambda, \quad 0+0=0$\\
	$a*0=a(0+0)=a*0+a*0$ прибавим к обеим частям противоположный элемент $(-a*0)$\\
	$(a*0)-(a*0)=-a*0+a*0+a*0 \Rightarrow 0=0_a*0\Rightarrow a*0=0$\\\\
	
	\underline{Пункт \ref{e_b}}: $-(-a)=a$\\
	$(-a)$ противоположный для $(a)$, $(a)$ - противоположный для $(-a)$\\\\
	
	\underline{Пункт \ref{e_c}}: $(-a)(b)=-(ab)$\\
	т.к. $-(ab)$ противоположный к $(ab)$, то для доказательства достаточно показать, что $-(a)b$ противоположен $(ab)$:\\
	$ab+(-a)b=(a+(-a))b=0*b=0$\\
	$a*(-b)=-(ab)$ - аналогично\\\\
	
	\underline{Пункт \ref{e_d}}: $(-a)(-b)=ab$\\
	$(-a)(-b)=-(a(-b))=-(-ab)=ab$\\\\
	
	\underline{Пункт \ref{e_e}}: $a(b-c)=ab-ac$\\
	$a(b-c)=a(b+(-c))=ab+a(-c)=ab+(-(ac))=ab-ac$\\\\
	
	\underline{Пункт \ref{e_f}}: $m(ab)=(ma)b=a(mb)$\\
	Для доказательства достаточно воспользоваться определением $n$-кратного элемента. Свойствами ассоциативности умножения пользоваться нельзя.\\
	\begin{enumerate}
		\item $m\in \NN \quad m(ab)=ab+ab+...+ab=(a+...+a)b=(ma)b$
		\item $m=0 \quad m(ab)=(ma)b\Rightarrow$ из свойства \ref{e_a}
		\item $m\in\ZZ \backslash \NN_0 \Rightarrow m=-n, \quad$ где $n\in \NN$\\
			$-n(ab)=(-na)b \qquad (-n)a=-(na) \quad$- по определению $n$-кратного элемента\\
			$-n(ab)=(-na)b, \quad$ так как $(-a)b=a(-b)=-ab$ 
	\end{enumerate}

	\underline{Пункт \ref{e_g}}: $(m_1a)(m_2b)=(m_1m_2)(ab)$\\
	$(m_1a)(m_2b)=(a+a+...+a)(b+b+...+b)=ab+ab+...+ab=m_1(m_2ab)$
	
\end{proof}





\end{document}
