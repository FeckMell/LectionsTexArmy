\documentclass[12pt,oneside]{article}
\usepackage[utf8]{inputenc}
\usepackage[english,russian]{babel}
\usepackage{enumerate}
\usepackage{indentfirst}
\usepackage{misccorr}
\usepackage{amsmath}
\usepackage{amssymb}
\usepackage{amsthm}
\usepackage{amsfonts}
\usepackage{mathtools}
\usepackage{color}
\usepackage{cmap}
\usepackage{titlesec}
\usepackage[dvips]{graphicx}
%\graphicspath{C:\Users\dm\Desktop\Диплом}
\usepackage{listings}




%\renewcommand{\abstractname}{}
\newtheorem{theorem}{Теорема}[section]
\newtheorem{hypoth}[theorem]{Гипотеза}
\newtheorem{lemma}[theorem]{Лемма}
\newtheorem{determ}[theorem]{Определение}
\newtheorem{naming}[theorem]{Обозначение}
\newtheorem{ffff}[theorem]{Замечание}
\newtheorem{sentence}[theorem]{Предложение}
\newtheorem{iwantthat}[theorem]{Утверждение}
\newtheorem{fromit}[theorem]{Следствие}
\theoremstyle{definition}
\newtheorem{example}[theorem]{Пример}
\newtheorem{task}[theorem]{Задача}
\setcounter{MaxMatrixCols}{50}
%\DeclarePairedDelimiter{\norm}{\lVert}{\rVert}


\begin{document}




\tableofcontents


%
%
%
%
%
\newpage
\section{Основные алгебраические структуры}\label{ch_basic_alg_st}
\subsection{Бинарные операции и их свойства}

\begin{determ}
Пусть $A$ - множество. Бинарной операцией на множестве $A$ называется отображение:
\begin{equation}
f:A^2 \rightarrow A \quad (f: A \times A \rightarrow A)
\end{equation}
\end{determ}

\begin{ffff}
	Если $f$ - бинарная операция на $A$ и пара $(a,b) \in A^2$, то образ пары $(a,b)$ при отображении $f$ называется значением операции $f$ на элементах $a$ и $b$ (результатом применения операции $f$ к элементам $a$ и $b$) и обозначается $f(a,b)$ или $afb$.
\end{ffff}

\begin{example}\label{bin_examples}
	Примеры бинарных операций
	\begin{enumerate}
		\item Сложение и усножение на множествах $N, N_0, Z, Q, R$.
		\item Вычитание на $Z, Q, R$ - определеноб $N, N_0$ - не определено.
		\item Пусть $f_1, f_2$ такие, что $f_1:(1,n)^2 \rightarrow (\overline{1,n})$, $f_2:(\overline{1,n})^2 \rightarrow(1,n)$ при этом
		\begin{equation}
			f_1(a,b) = \max\{a,b\},\quad f_2(a,b) = \min\{a,b\}
		\end{equation}\\
		Так как $\forall a,b \in \overline{1,n}$ $\max$ и $\min$ однозначно определены и содержатся во множестве от $\overline{1,n}$, то отображения $f_1, f_2$ являются бинарными операциями на множестве $\overline{1,n}$.
		\item $M$ - множество, $P(M)$ - множество всех подмножеств, тогда пересечение и объединение $\forall A,B \in P(M)$ является бинарными операциями на множестве $P(M)$. 
	\end{enumerate}
\end{example}

\begin{determ}
	Бинарная операция $*$ на множестве $M$ называется ассоциативной, если $\forall a,b \in M$ выполняется условие $(a*b)*c = a*(b*c)$.\\
	Примерами ассоциативных операций могут служить бинарные операции из примера  \ref{bin_examples}.
\end{determ}

\begin{determ}
	Бинарная операция $*$ на множестве $M$ называется коммутативной, если $\forall a,b \in M$ выполняется условие
	\begin{equation}
		a*b=b*a
	\end{equation}
\end{determ}

\begin{example}
	Примеры коммутативных и некоммутативных операций:
	\begin{enumerate}
		\item Коммутативные - пункты $1, 3, 4$ из \ref{bin_examples}.
		\item Некоммутативная - пункт $2$ из \ref{bin_examples}, декартово произведение, композиция.
	\end{enumerate}
\end{example}

\begin{ffff}
	Для отдельных элементов $a,b \in M$ равенство $a*b=b*a$ может выполняться в том случае, если операция $*$ не коммутативна. Такие элементы называются перестановочными (коммутирующими) друг с другом
\end{ffff}

\begin{example}
	$a=0, b=0: \quad a-b=b-a$
\end{example}

\begin{ffff}
	Свойства ассоциативности и коммутативности операции независимы. пример коммутативной но не ассоциативной операции:
	\begin{equation}
	a*b=\frac{a+b}{2}
	\end{equation}
\end{ffff}

\begin{determ}
	Бинарная операция $*$ на множестве $M$ называется леводистрибутивной(праводистрибутивной) относительно операции $o$ если $\forall$ $a,b,c \in M$ выполнено условие:
	\begin{gather}
	a*(b \circ c) = (a*b)\circ(a*c) \qquad\mbox{ - леводистрибутивная}\\
	(b \circ c)*a = (b*a)\circ(c*a) \qquad\mbox{ - праводистрибутивная}
	\end{gather}
	если выполняются оба этих равенства, то говорят, что $*$ дистрибутивна относительно операции $\circ$. Например умножение дистрибутивна к сложению.
\end{determ}


\subsection{Алгебраические структуры с одной бинарной операцией}
\begin{determ}
	Алгебраической структурой (алгеброй) называется множество с системой операций.
\end{determ}

\begin{determ}
	Множество $G$ с одной бинарной операцией называют группоидом, обозначают $(G,*)$.
\end{determ}

\begin{ffff}
	Из определения группоида следует, что если множество $G$ - конечно, то правило по которому можно найти значение операции $*$. $\forall a,b \in G$ , можно записать в таблицу $G=\{a_1, \dots, a_n\}$ - т.к. $G$  - конечно.\\
	$\begin{pmatrix}
		* & a_1 & \dots & a_j & \dots & a_n \\
		a_1& a_1*a_1 & \dots & a_1*a_j & \dots & a_1*a_n \\
		\dots &\dots &\dots &\dots &\dots &\dots \\
		a_i& a_i*a_1 & \dots & a_i*a_j & \dots & a_i*a_n \\
		\dots &\dots &\dots &\dots &\dots &\dots \\
		a_n& a_n*a_1 & \dots & a_n*a_j & \dots & a_n*a_n \\
		
	\end{pmatrix}$
\end{ffff}

\begin{determ}
	Пусть $G_1 \subset G$, $\exists (G,*),$ $G_1$ называют замкнутым относительно операции $*$, если выполнены условия $\forall a,b \in G: ab\in G_1$.\\
	В этом случае группоид $(G_1,*)$ называют подгруппоидом группоида $(G,*)$.
\end{determ}

\begin{determ}
	Элемент $\Lambda$ группоида $(G,*)$ называют нейтральным, если $\forall a \in G$ выполнено:\\
	\begin{equation}
	\Lambda*a=a
	\end{equation}
\end{determ}







\end{document}
